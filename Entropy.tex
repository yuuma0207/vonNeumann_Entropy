\input{~/macro.tex}
\title{von Neumannエントロピーを元にした熱力学第二法則の導出}
%\author{20B01392 松本侑真}
\date{\today}
\begin{document}
\maketitle
\begin{abstract}
	系$S$と熱浴$B$が接している状況を考える。熱浴は温度$\beta = (k_{\text{B}}T)^{-1}$であり、系に$Q$の熱を与えるとする。
	このとき、系$S$のエントロピー変化$k_{\text{B}}\varDelta S_S$と熱浴のエントロピー変化$Q/T$の和は必ず正になるというのが熱力学第2法則である:
	\begin{equation*}
		\varDelta S_S + \beta Q \geq 0\;。
	\end{equation*}
	以下では、平衡熱力学に基づかないセットアップからスタートして、熱力学第二法則を導出する。
\end{abstract}
\tableofcontents
\section{セットアップ}
系$S$と熱浴$B$が接しているとき、全系のHamiltonian$\hat{H}$は
\begin{equation}
	\hat{H} = \hat{H}_S + \hat{H}_B + \hat{H}_I
\end{equation}
と表される。ここで、$\hat{H}_S,\,\hat{H}_B$はそれぞれ系$S$と熱浴$B$が独立して存在する場合のHamiltonianであり、$\hat{H}_I$は系$S$と熱浴$B$の相互作用Hamiltonianである。
次に、密度行列$\hat{\rho}$が与えられた際に定義されるvon Neumannエントロピーを導入する:
\begin{equation}
	S(\hat{\rho}) \coloneqq -\Trace\qty(\hat{\rho}\ln\hat{\rho})\;。
\end{equation}
密度行列とは、系全体を張る状態ベクトルの集合$\ket{\phi_0},\,\ket{\phi_1},\,\ldots\ket{\phi_{N-1}}$と、それぞれの状態が実現する確率$p_0,\,p_1,\,\ldots,\,p_{N-1}\quad\qty(\sum p_i = 1)$が与えられたときに、
\begin{equation}
	\hat{\rho} = \sum_{i=0}^{N-1}p_i\ket{\phi_i}\bra{\phi_i}
\end{equation}
と定義される。正規直交基底$\ket{0},\,\ket{1},\,\ldots,\,\ket{N-1}$を用いて$\Trace$を計算できることより、密度行列は
\begin{equation}
	\Trace{\hat{\rho}} = \sum_n\bra{n}\hat{\rho}\ket{n} = \sum_{i=0}^{N-1}p_i = 1
\end{equation}
となる性質を持つ。さらに、von Neumannエントロピーは形式的にShannonエントロピーと一致する:
\begin{equation}
	S(\hat{\rho}) = -\sum_{i=0}^{N-1} p_i\ln{p_i}\;\text{($=$Shannonエントロピー)}\;。
\end{equation}
時刻$t$における全系の密度行列を$\hat{\rho}(t)$とおくと、密度行列の初期状態は$S$と$B$に相関がない状態で表すことができる:
\begin{equation}
	\hat{\rho}(0) = \hat{\rho}_S(0)\otimes\hat{\rho}_B(0) = \hat{\rho}_S(0)\otimes\hat{\rho}^{can}_B\;。
\end{equation}
ただし、熱浴の初期状態はカノニカル分布での熱平衡状態を実現していると仮定した:
\begin{equation}
	\hat{\rho}_B^{can} = \frac{e^{-\beta\hat{H}_B}}{Z}\;,\quad Z = \sum_{n}e^{-\beta E_n}\;。
\end{equation}

\section{証明}
初期状態でのvon Neumannエントロピーは、
\begin{align}
	S(\hat{\rho}(0)) & = S(\hat{\rho}_S(0)) + S(\hat{\rho}_B^{can}) = S(\hat{\rho}_S(0)) - \Trace{\hat{\rho}_B^{can}\ln\frac{e^{-\beta\hat{H}_B}}{Z}} \notag                    \\
	                 & = S(\hat{\rho}_S(0)) - \Trace{\hat{\rho}_B^{can}\qty(-\beta\hat{H}_B - \ln{Z})} = S(\hat{\rho}_S(0)) + \beta\Trace{\hat{\rho}_B^{can}\hat{H}_B} + \ln{Z}
	\label{eq:shoki}
\end{align}
と計算できる。
\footnote{密度行列の$\Trace$が1になることを用いた:
	\begin{equation}
		\Trace\hat{\rho} = \sum_i p_i = 1\;。
	\end{equation}
}
任意の時刻$t$におけるエントロピーは、任意の密度行列$\hat{\rho},\,\hat{\sigma}$に対して成立する不等式
\begin{equation}
	\Trace{\hat{\rho}\ln\hat{\rho}} \geq \Trace{\hat{\rho}\ln\hat{\sigma}}
\end{equation}
を用いると、$\hat{\rho}(t) = \hat{\rho}_S(t)\otimes\hat{\rho}_B(t)$として、
\begin{align}
	S(\hat{\rho}(t)) = -\Trace{\hat{\rho}(t)\ln\hat{\rho}(t)} & \leq -\Trace{\hat{\rho}(t)\ln\hat{\rho}_S(t)\otimes\hat{\rho}_B^{can}} \notag    \\
	                                                          & =S(\hat{\rho}_S(t)) + \beta\Trace{\hat{\rho}_B(t)\hat{H}_B} + \ln{Z}\label{eq:S}
\end{align}
となる。また、全系はユニタリ発展をしているため、
\begin{align}
	S(\hat{\rho}(t)) & = S(U\hat{\rho}(0)U^\dagger) = -\Trace(U\hat{\rho}(0)\ln\hat{\rho}(0)U^\dagger) \notag                                      \\
	                 & = -\Trace(\hat{\rho}(0)\ln\hat{\rho}(0)U^\dagger U) = -\Trace(\hat{\rho}(0)\ln\hat{\rho}(0)) = S(\hat{\rho}(0))\label{eq:U}
\end{align}
となる。式\eqref{eq:shoki}、式\eqref{eq:S}、式\eqref{eq:U}を用いると、、系$S$のエントロピー変化を$\varDelta S_S \coloneqq S(\hat{\rho}_S(t)) - S(\hat{\rho}_S(0))$として、
\begin{equation}
	S(\hat{\rho}_S(t)) + \beta\Trace{\hat{\rho}_B(t)\hat{H}_B} + \ln{Z} - S(\hat{\rho}(t)) = \varDelta S_S + \beta\Trace\qty[\qty(\hat{\rho}_B(t) - \hat{\rho}_B(0))\hat{H}_B]\geq 0
\end{equation}
が成立する。熱浴$B$のエネルギー変化を$\varDelta E$、始状態から時刻$t$までに熱浴$B$から系$S$に流入した熱を$Q$とおくと、
$\Trace\qty[\qty(\hat{\rho}_B(t) - \hat{\rho}_B(0))\hat{H}_B] =\Trace{\hat{\rho}_B(t)\hat{H}_B} - \Trace{\hat{\rho}_B(0)\hat{H}_B} =  -\varDelta E_B = Q$が成立することより、
熱力学第二法則が成立することが示された:
\begin{equation}
	\varDelta S_S + \beta Q\geq 0\;。
\end{equation}
\section{必要な定理一覧}

\begin{tcolorbox}[
		colback = white,
		colframe = green!35!black,
		fonttitle = \bfseries]
	\begin{theorem}[エルミート演算子を引数にもつ関数について]
		$\hat{A}$をヒルベルト空間$\mathcal{H}$上のエルミート演算子、$f:x\in \mathbb{R} \mapsto f(x)\in \mathbb{R}$を実数値関数とするとき、スペクトル分解
		\begin{equation}
			\hat{A} = \sum_{k=1}^m a_k\ket{\phi_k}
		\end{equation}
		を用いて$f(\hat{A})$を次のように定義する:
		\begin{equation}
			f(\hat{A}) \coloneqq \sum_{k=1}^{m} f(a_k)\ket{\phi_k}\;。
		\end{equation}
		ただし、$\hat{A}$の固有値は$f$の定義域に入っているとする。
		\label{th:kansu}
	\end{theorem}
\end{tcolorbox}
\vskip\baselineskip
\begin{tcolorbox}[
		colback = white,
		colframe = green!35!black,
		fonttitle = \bfseries]
	\begin{theorem}[カノニカル分布の密度行列について]
		系がカノニカル分布で熱平衡状態を達成しているときの密度行列$\hat{\rho}^{can}$を考える。系の状態$\ket{n}$の実現確率は$p_n =e^{-\beta E_n}/Z,\,Z = \sum_n e^{-\beta E_n}$と与えられる。
		$\hat{H}\ket{n} = E_n\ket{n}$が成立していることと定理\ref{th:kansu}から、密度行列$\hat{\rho}^{can}$は
		\begin{equation}
			\hat{\rho}^{can} = \sum_n p_n\ket{n}\bra{n} = \sum_n\ket{n}\frac{e^{-\beta E_n}}{Z}\bra{n} = \frac{e^{-\beta\hat{H}}}{Z}
		\end{equation}
		と求まる。
	\end{theorem}
\end{tcolorbox}
\vskip\baselineskip
\begin{tcolorbox}[
		colback = white,
		colframe = green!35!black,
		fonttitle = \bfseries]
	\begin{theorem}[von Neumannエントロピーの加法性]
		von Neumannエントロピーに関して加法性
		\begin{equation}
			S(\hat{\rho}_A\otimes\hat{\rho}_B) = S(\hat{\rho}_A) + S(\hat{\rho}_B)
		\end{equation}
		が成立する。以下ではこれを示す。

		まず、$\hat{\rho}_A,\,\hat{\rho}_B$のスペクトル分解を考える:
		\begin{equation}
			\hat{\rho}_A = \sum_n \lambda_{n,A}\ket{n}_A,\,\hat{\rho}_B = \sum_n\lambda_{n,B}\ket{n}_B\;。
		\end{equation}
		これより
		\begin{equation}
			\hat{\rho}_A\otimes\hat{\rho}_B = \sum_{n,\,n'}\lambda_{n,A}\lambda_{n',B}\ket{n}_A\otimes\ket{n'}_B
		\end{equation}
		が成立するため、
		\begin{align}
			\ln(\hat{\rho}_A\otimes\hat{\rho}_B) & = \sum_{n,\,n'}\ln(\lambda_{n,A}\lambda_{n',B})\ket{n}_A\otimes\ket{n'}_B \notag                                          \\
			                                     & = \sum_n \ln\lambda_{n,A}\ket{n}_A\otimes\sum_{n'}\ket{n'}_B + \sum_n\ket{n}_A\otimes\sum_{n'}\ln\lambda_{n',B}\ket{n'}_B
			= \hat{\rho}_A\otimes I_B + I_A\otimes\hat{\rho}_B
		\end{align}
		となる。したがって、von Neumannエントロピーの加法性
		\begin{align}
			S(\hat{\rho}_A\otimes\hat{\rho}_B) & = -\Trace{\qty[\qty(\hat{\rho}_A\otimes\hat{\rho}_B)\ln(\hat{\rho}_A\otimes\hat{\rho}_B)]} \notag                                                              \\
			                                   & = -\Trace\qty(\hat{\rho}_A\ln\hat{\rho}_A\otimes \hat{\rho}_B) -\Trace\qty(\hat{\rho}_A\otimes\hat{\rho}_B\ln\hat{\rho}_B) = S(\hat{\rho}_A) + S(\hat{\rho}_B)
		\end{align}
		を得る。
	\end{theorem}
\end{tcolorbox}

\vskip\baselineskip
\begin{tcolorbox}[
		colback = white,
		colframe = green!35!black,
		fonttitle = \bfseries]
	\begin{theorem}[物理量の期待値は密度行列を用いて計算できる]
		物理量$A$に対応するエルミート演算子を$\hat{A}$とすると、$A$の期待値$\expval{A}$は
		\begin{equation}
			\expval{A} = \Trace{\hat{\rho}\hat{A}}
		\end{equation}
		と表すことができる。系の正規直交基底$\ket{n}$を用いると、実際に右辺は次のように計算できる:
		\begin{equation}
			\Trace{\hat{\rho}\hat{A}} = \sum_n\bra{n}\hat{\rho}\hat{A}\ket{n} = \sum_{n,\,i}p_i\bra{n}\ket{i}\bra{i}\hat{A}\ket{n} = \sum_{n,\,i}p_i\delta_{n,i}a_n = \sum_ip_ia_i\;。
		\end{equation}
		$a_i$は$\hat{A}$の固有値であり、物理量$A$の観測値である。そのため、$\sum_ip_ia_i = \expval{A}$となる。

	\end{theorem}
\end{tcolorbox}
\vskip\baselineskip
\begin{tcolorbox}[
		colback = white,
		colframe = green!35!black,
		fonttitle = \bfseries]
	\begin{theorem}[系の時間発展は密度行列のユニタリ発展に押し付けることができる]
		初期状態での系の状態ベクトル$\ket{\psi(0)}$が与えられているとする。このとき、任意の時刻$t$における系の状態ベクトル$\ket{\psi(t)}$は、ユニタリ行列$U(t)$を用いて
		\begin{equation}
			\ket{\psi(t)} = U(t)\ket{\psi(0)}
		\end{equation}
		と表される。なお、$U(t) = \exp\qty(\frac{\hat{H}(t)}{i\hbar})$である。$\ket{\psi(t)}$が時間に依存するSchr\"{o}dinger方程式を満たすことは簡単にわかるであろう。
		ここで、ユニタリ発展をする系においては、Hamiltonianのとある固有状態を実現する確率が時刻に依らないことを用いると、時刻$t$における密度行列$\hat{\rho}(t)$は
		\begin{equation}
			\hat{\rho}(t) = \sum_ip_i\ket{\phi_i(t)}\bra{\phi_i(t)} = \sum_ip_iU\ket{\phi_i(0)}\bra{\phi_i(0)}U^\dagger = U\hat{\rho}(0)U^\dagger
		\end{equation}
		となる。すなわち、系がユニタリな時間発展をする(時間に依存するShr\"{o}dinger方程式に従って時間発展する)際には、任意の時刻$t$の密度行列を初期状態の密度行列で表すことができる。
		なお、$\ket{\phi_i(t)}$は$\hat{H}(t)$の固有状態とした。
	\end{theorem}
\end{tcolorbox}

\end{document}